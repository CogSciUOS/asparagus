%----------------------------------------------------------------------------------------
%	PREPROCESSING PIPELINE AND DATASET CREATION
%----------------------------------------------------------------------------------------
\section{Preparing the dataset}

Different preprocessing steps of the data and deciding how the data should/will look like. The label app is introduced and the process of manually labeling a part of the data, with a description of the criteria. Finally, the dataset is introduced.


\subsection{Preprocessing steps}

First approach to create a dataset (layout) and data augmentation to generate more samples.


\subsubsection{Automatic feature extraction}

We (tried to) create scripts for:

\begin{itemize}
\item background removal of the collected images
\item automatic feature extraction pipeline (including the decision that we first try to sort for features and not labels) for

\begin{itemize}
\item rust
\item bent
\item etc. ...
\end{itemize}

\end{itemize}


\subsubsection{Preparation for manual feature extraction}

Preparing the images for manual classification to create more labeled data.

\begin{itemize}
\item sorting the pictures in the grid
\item have 3 pictures per asparagus spear
\end{itemize}


\subsection{The hand-label app}

Introduction to the script created for manual sorting. Fusion of the feature extraction scripts.

\begin{itemize}
\item What is it?
\item Why did we need it? What was the idea behind it?
\item How does it work? (keep short! it's only the introduction)
\end{itemize}

Do not explain in length here but rather give an idea and refer to README's and to code in GitHub whenever possible.


\subsubsection{How to install}

Installation of the app.

\begin{itemize}
\item environment setup
\item mount points
\item problems we ran into
\item etc.
\end{itemize}


\subsubsection{Operating instructions}

User manual for the app and introduction to its graphical user interface.

\begin{itemize}
\item What can you find where? (include one example picture)
\item Step-by-step guideline through

\begin{enumerate}
\item loading pictures,
\item creating a .csv file,
\item and how to sort one picture.
\end{enumerate}

\end{itemize}


\subsubsection{Performance}

Results and general performance of the app.

\begin{itemize}
\item How well did the feature extraction work?
\item How much features had to be labeled by hand?
\item What is the output of the app?
\end{itemize}


\subsection{Manual labeling}

The process and outcome of the hand-labeling of the asparagus images.


\subsubsection{Sorting criteria}

The criteria explained in detail for the hand-labeling of the features with the app.

\begin{itemize}
\item The single sorting criteria for rust, bent, flower, etc. (including example pictures)
\item What are expected difficulties we might encounter?
\end{itemize}


\subsubsection{Sorting outcome}

The process and the results of the sorting. 

\begin{enumerate}
\item How much did we sort?
\item How well did the sorting work in general?
\begin{enumerate}
\item i.e., was it easy to sort?
\item how long did it take?
\item what problems were encountered?
\end{enumerate}
\item How accurately did we sort as a group?
\begin{enumerate}
\item i.e., Kappa Agreement
\end{enumerate}
\end{enumerate}


\subsection{The asparagus dataset}

Any information on the data we worked with for our approaches.


\subsubsection{Different datasets}

Structural information on the datasets.

\begin{itemize}
\item What do they look like?
\item How big are they (labeled vs unlabeled samples)?
\item Which were criteria for throwing out data?
\item maybe an overview picture with all relevant information on one glance
\end{itemize}


\subsubsection{Challenges}

Problems and challenges during the creation of the datasets.

\begin{itemize}
\item What were the challenges in creating a general dataset?
\item What were challenges in general?
\item How well could we work with the datasets?
\item What was used as training data, validation data, and test data?
\end{itemize}
