Eye tracking experiments rely heavily on good data quality of eye trackers. 
Unfortunately, often only spatial accuracy and precision values are available from the manufacturers.
These two values alone are not sufficient to reasonably benchmark an eye tracker:
The values will deteriorate drastically during an experimental session due to head movements, changing illumination or calibration decay in general. In addition, different experimental paradigms allow to analyze different types of eye movements which cannot be evaluated by spatial accuracy or precision, for instance smooth pursuit movements, blinks or microsaccades.
To obtain a more comprehensive description of properties, we developed an extensive eye tracking test battery.
In 10 different tasks, we evaluated eye tracking related measures such as: the decay of accuracy, pupil dilation, smooth pursuit movement, microsaccade detection, blink detection, and the influence of head motion.
For some measures true theoretical values exist, for others, a relative comparison to a gold standard eye tracker is needed.
Therefore, we collected our gaze data simultaneously from a gold standard remote EyeLink~1000 (\SI{500}{\hertz}) eye tracker and the mobile Pupil Labs glasses (\SI{120}{\hertz} -- \SI{240}{\hertz}).