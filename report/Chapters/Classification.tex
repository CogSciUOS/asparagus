%----------------------------------------------------------------------------------------
%	CLASSIFICATION OF DATA VIA DIFFERENT METHODS
%----------------------------------------------------------------------------------------
\section{Classification}

Classification of the data using different approaches.

\subsection{Supervised learning}

Convolutional neural networks created to sort the samples for their features. \\
\\
INTRODUCTION \\
... \\
\\
GENERAL BACKGROUND \\
... \\
\\
IDEAS FOR MODELS \\
Ideas for models to use for supervised learning. \\
\\
OUTRO \\
Transfer to the specific model ideas that we worked on for supervised learning.

\subsubsection{Single-label classification}

Single-label approach, where the network sorts for one feature. \\
\\
\\
\\
INTRODUCTION \\
\\ 
- 2-3 introductory sentences ( e.g., One supervised approach for training a model is xxx. It uses xxx and works like xxx.) \\
\\
- What approach did I choose? (short) \\
\\
- Why did I choose it? (What are the advantages/disadvantages of the model/approach? What results do I expect?) (short) \\
\\
\\
\\
BACKGROUND \\
\\
- General background of the approach (all the literature, ideas, inspirations, theory, etc ...) (medium) \\
\\
\\
\\
ACTUAL MODEL STRUCTURE (detailed) \\
\\
- Overview of model design (including picture(s) of structure) \\
\\
- Challenges (What was tested/changed during the working process? Which obstacles did occur?) \\
\\
- What are potentials/risks of interpreting the results? (e.g. when working with my model the potential for better results was using features instead of classes while the risk could be … ) \\
\\
\\
\\
RESULTS (detailed) \\
\\
- How good did my model predict? \\
\\
\\
\\
DISCUSSION (detailed) \\
\\
- Why did it produce the results it produced? \\
\\
- What can I interpret from my results? \\
\\
- What could not be done/expected? \\
\\
- What is still missing now? What would I do when continuing to work on the model? \\
\\

\subsubsection{Multi-label classification}

Multi-label approach, where the network labels everything at the same time. \\
\\
\\
\\
INTRODUCTION \\
\\ 
- 2-3 introductory sentences ( e.g., One supervised approach for training a model is xxx. It uses xxx and works like xxx.) \\
\\
- What approach did I choose? (short) \\
\\
- Why did I choose it? (What are the advantages/disadvantages of the model/approach? What results do I expect?) (short) \\
\\
\\
\\
BACKGROUND \\
\\
- General background of the approach (all the literature, ideas, inspirations, theory, etc ...) (medium) \\
\\
\\
\\
ACTUAL MODEL STRUCTURE (detailed) \\
\\
- Overview of model design (including picture(s) of structure) \\
\\
- Challenges (What was tested/changed during the working process? Which obstacles did occur?) \\
\\
- What are potentials/risks of interpreting the results? (e.g. when working with my model the potential for better results was using features instead of classes while the risk could be … ) \\
\\
\\
\\
RESULTS (detailed) \\
\\
- How good did my model predict? \\
\\
\\
\\
DISCUSSION (detailed) \\
\\
- Why did it produce the results it produced? \\
\\
- What can I interpret from my results? \\
\\
- What could not be done/expected? \\
\\
- What is still missing now? What would I do when continuing to work on the model? \\
\\


\subsection{Semi-supervised learning}

Train a network on labeled and unlabeled samples. \\
\\
INTRODUCTION \\
... \\
\\
GENERAL BACKGROUND \\
... \\
\\
IDEAS FOR MODELS \\
Ideas for models to use for semi-supervised learning. \\
\\
OUTRO \\
Transfer to the specific model ideas that we worked on for semi-supervised learning. \\
\\

\subsubsection{Autoencoder}
\\
INTRODUCTION \\
\\ 
- 2-3 introductory sentences ( e.g., One supervised approach for training a model is xxx. It uses xxx and works like xxx.) \\
\\
- What approach did I choose? (short) \\
\\
- Why did I choose it? (What are the advantages/disadvantages of the model/approach? What results do I expect?) (short) \\
\\
\\
\\
BACKGROUND \\
\\
- General background of the approach (all the literature, ideas, inspirations, theory, etc ...) (medium) \\
\\
\\
\\
ACTUAL MODEL STRUCTURE (detailed) \\
\\
- Overview of model design (including picture(s) of structure) \\
\\
- Challenges (What was tested/changed during the working process? Which obstacles did occur?) \\
\\
- What are potentials/risks of interpreting the results? (e.g. when working with my model the potential for better results was using features instead of classes while the risk could be … ) \\
\\
\\
\\
RESULTS (detailed) \\
\\
- How good did my model predict? \\
\\
\\
\\
DISCUSSION (detailed) \\
\\
- Why did it produce the results it produced? \\
\\
- What can I interpret from my results? \\
\\
- What could not be done/expected? \\
\\
- What is still missing now? What would I do when continuing to work on the model? \\
\\

\subsection{Unsupervised learning}

Unsupervised approach to sort the asparagus. \\
\\
INTRODUCTION \\
... \\
\\
GENERAL BACKGROUND \\
... \\
\\
IDEAS FOR MODELS \\
Ideas for models to use for unsupervised learning. \\
\\
OUTRO \\
Transfer to the specific model ideas that we worked on for unsupervised learning. \\
\\
\subsubsection{Principal component analysis}
\\
INTRODUCTION \\
\\ 
- 2-3 introductory sentences ( e.g., One supervised approach for training a model is xxx. It uses xxx and works like xxx.) \\
\\
- What approach did I choose? (short) \\
\\
- Why did I choose it? (What are the advantages/disadvantages of the model/approach? What results do I expect?) (short) \\
\\
\\
\\
BACKGROUND \\
\\
- General background of the approach (all the literature, ideas, inspirations, theory, etc ...) (medium) \\
\\
\\
\\
ACTUAL MODEL STRUCTURE (detailed) \\
\\
- Overview of model design (including picture(s) of structure) \\
\\
- Challenges (What was tested/changed during the working process? Which obstacles did occur?) \\
\\
- What are potentials/risks of interpreting the results? (e.g. when working with my model the potential for better results was using features instead of classes while the risk could be … ) \\
\\
\\
\\
RESULTS (detailed) \\
\\
- How good did my model predict? \\
\\
\\
\\
DISCUSSION (detailed) \\
\\
- Why did it produce the results it produced? \\
\\
- What can I interpret from my results? \\
\\
- What could not be done/expected? \\
\\
- What is still missing now? What would I do when continuing to work on the model? \\
\\

\subsection{From feature to label}

Linking the features to their designated label. \\
Combining the outcome of multiple networks for further processing.