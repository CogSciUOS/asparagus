%----------------------------------------------------------------------------------------
%	CLASSIFICATION OF DATA VIA DIFFERENT METHODS
%----------------------------------------------------------------------------------------
\section{Classification}

Classification of the data using different approaches. \\

\\

Given the structure of our dataset, namely image data with a sub dataset with corresponding class labels, and a sub dataset with corresponding feature labels, different machine learning and computer vision methods were chosen, to tackle the problem of image classification. \\

Image classification refers to the method of identifying to which category an image belongs to, according to its visual information.  Classification problems can be divided into three different types: binary, multi-class and multi-label. Whereas a binary classification only distinguishes between two different classes and therefore classifies an image into one of the two classes, a multi-class classification distinguishes between multiple exclusive classes. A multi-label classification also works for multiple classes. However, a single image can belong to none, one, several or all of the classes. Those classes can be seen as a feature vector for each image. Each feature can be present or not, independently of the other features. ( vielleicht hier zitieren, seite muss noch rausgesucht werden: Har-Peled, S., Roth, D., Zimak, D. (2003) "Constraint Classification for Multiclass Classification and Ranking." In: Becker, B., Thrun, S., Obermayer, K. (Eds) Advances in Neural Information Processing Systems 15: Proceedings of the 2002 Conference, MIT Press. ISBN 0-262-02550-7) \\

Each of those classification methods comes with certain advantages and implications.
There exist many classification methods which have been developed for binary classification problems, but less methods are suited for multi-label or multi-class classification. Therefore, the latter often work as a combination of binary classifiers. Moreover, multi-class and multi-label classification have the difficulty of sparser labels. \\

Binary classification can be used to decide whether a certain feature is present at one certain asparagus piece, or not. This is helpful for a first inspection of the data, but does not enable a full classification of one image into one of 13 classes, which are currently sorted at the Spargelhof Gut Hosterfeld. Multi-class classification solves this problem. It is ferly easy to apply this classification type on the prelabeled images, but increasingly difficult for the semi-supervised and unsupervised approaches. While it enables a clear identification of class belonging, it also does not enable to train variability within classes. As the class id results from a combination of the presence of certain features, and not others, it is therefore also reasonable to go for a multi-label classification approach. \\

Moreover, there are different methods on how to approach image classification. Those can be divided into three main groups: supervised learning, semi supervised learning and unsupervised learning. Additionally to classical computer vision based approaches, our study group investigated several neural network approaches. \\

During our group work, algorithms of all three different classification types (binary, multiclass and multilabel) as well as of all three learning types (supervised, semi-supervised and unsupervised) were applied for different working steps and different purposes. \\

In the long run, an integrated model was aimed which predicts all features of a single asparagus piece, and from which an additional class can be inferred. However, as intermediate steps towards that goal, the focus was to optimize models on identifying the presence of single features. Besides that, we only investigated a few multi-label classification tasks. \\

The following chapter aims to give a general background of the different approaches chosen for our image classification problem, as well as a detailed overview of the concrete implementations of the models and the mechanisms of their hyperparameters.
All algorithms were implemented in Python.  \\



\subsection{Supervised learning}

Convolutional neural networks created to sort the samples for their features. \\
\\
INTRODUCTION \\
... \\
\\
GENERAL BACKGROUND \\
... \\
\\
IDEAS FOR MODELS \\
Ideas for models to use for supervised learning. \\
\\
OUTRO \\
Transfer to the specific model ideas that we worked on for supervised learning.

\subsubsection{Single-label classification}

Single-label approach, where the network sorts for one feature. \\
\\
\\
\\
INTRODUCTION \\
\\
- 2-3 introductory sentences ( e.g., One supervised approach for training a model is xxx. It uses xxx and works like xxx.) \\
\\
- What approach did I choose? (short) \\
\\
- Why did I choose it? (What are the advantages/disadvantages of the model/approach? What results do I expect?) (short) \\
\\
\\
\\
BACKGROUND \\
\\
- General background of the approach (all the literature, ideas, inspirations, theory, etc ...) (medium) \\
\\
\\
\\
ACTUAL MODEL STRUCTURE (detailed) \\
\\
- Overview of model design (including picture(s) of structure) \\
\\
- Challenges (What was tested/changed during the working process? Which obstacles did occur?) \\
\\
- What are potentials/risks of interpreting the results? (e.g. when working with my model the potential for better results was using features instead of classes while the risk could be … ) \\
\\
\\
\\
RESULTS (detailed) \\
\\
- How good did my model predict? \\
\\
\\
\\
DISCUSSION (detailed) \\
\\
- Why did it produce the results it produced? \\
\\
- What can I interpret from my results? \\
\\
- What could not be done/expected? \\
\\
- What is still missing now? What would I do when continuing to work on the model? \\
\\

\subsubsection{Multi-label classification}

Multi-label approach, where the network labels everything at the same time. \\
\\
\\
\\
INTRODUCTION \\
\\
- 2-3 introductory sentences ( e.g., One supervised approach for training a model is xxx. It uses xxx and works like xxx.) \\
\\
- What approach did I choose? (short) \\
\\
- Why did I choose it? (What are the advantages/disadvantages of the model/approach? What results do I expect?) (short) \\
\\
\\
\\
BACKGROUND \\
\\
- General background of the approach (all the literature, ideas, inspirations, theory, etc ...) (medium) \\
\\
\\
\\
ACTUAL MODEL STRUCTURE (detailed) \\
\\
- Overview of model design (including picture(s) of structure) \\
\\
- Challenges (What was tested/changed during the working process? Which obstacles did occur?) \\
\\
- What are potentials/risks of interpreting the results? (e.g. when working with my model the potential for better results was using features instead of classes while the risk could be … ) \\
\\
\\
\\
RESULTS (detailed) \\
\\
- How good did my model predict? \\
\\
\\
\\
DISCUSSION (detailed) \\
\\
- Why did it produce the results it produced? \\
\\
- What can I interpret from my results? \\
\\
- What could not be done/expected? \\
\\
- What is still missing now? What would I do when continuing to work on the model? \\
\\


\subsection{Semi-supervised learning}

Train a network on labeled and unlabeled samples. \\
\\
INTRODUCTION \\
... \\
\\
GENERAL BACKGROUND \\
... \\
\\
IDEAS FOR MODELS \\
Ideas for models to use for semi-supervised learning. \\
\\
OUTRO \\
Transfer to the specific model ideas that we worked on for semi-supervised learning. \\
\\

\subsubsection{Autoencoder}
\\
INTRODUCTION \\
\\
- 2-3 introductory sentences ( e.g., One supervised approach for training a model is xxx. It uses xxx and works like xxx.) \\
\\
- What approach did I choose? (short) \\
\\
- Why did I choose it? (What are the advantages/disadvantages of the model/approach? What results do I expect?) (short) \\
\\
\\
\\
BACKGROUND \\
\\
- General background of the approach (all the literature, ideas, inspirations, theory, etc ...) (medium) \\
\\
\\
\\
ACTUAL MODEL STRUCTURE (detailed) \\
\\
- Overview of model design (including picture(s) of structure) \\
\\
- Challenges (What was tested/changed during the working process? Which obstacles did occur?) \\
\\
- What are potentials/risks of interpreting the results? (e.g. when working with my model the potential for better results was using features instead of classes while the risk could be … ) \\
\\
\\
\\
RESULTS (detailed) \\
\\
- How good did my model predict? \\
\\
\\
\\
DISCUSSION (detailed) \\
\\
- Why did it produce the results it produced? \\
\\
- What can I interpret from my results? \\
\\
- What could not be done/expected? \\
\\
- What is still missing now? What would I do when continuing to work on the model? \\
\\

\subsection{Unsupervised learning}

Unsupervised approach to sort the asparagus. \\
\\
INTRODUCTION \\
... \\
\\
GENERAL BACKGROUND \\
... \\
\\
IDEAS FOR MODELS \\
Ideas for models to use for unsupervised learning. \\
\\
OUTRO \\
Transfer to the specific model ideas that we worked on for unsupervised learning. \\
\\
\subsubsection{Principal component analysis}
\\
INTRODUCTION \\
\\
- 2-3 introductory sentences ( e.g., One supervised approach for training a model is xxx. It uses xxx and works like xxx.) \\
\\
- What approach did I choose? (short) \\
\\
- Why did I choose it? (What are the advantages/disadvantages of the model/approach? What results do I expect?) (short) \\
\\
\\
\\
BACKGROUND \\
\\
- General background of the approach (all the literature, ideas, inspirations, theory, etc ...) (medium) \\
\\
\\
\\
ACTUAL MODEL STRUCTURE (detailed) \\
\\
- Overview of model design (including picture(s) of structure) \\
\\
- Challenges (What was tested/changed during the working process? Which obstacles did occur?) \\
\\
- What are potentials/risks of interpreting the results? (e.g. when working with my model the potential for better results was using features instead of classes while the risk could be … ) \\
\\
\\
\\
RESULTS (detailed) \\
\\
- How good did my model predict? \\
\\
\\
\\
DISCUSSION (detailed) \\
\\
- Why did it produce the results it produced? \\
\\
- What can I interpret from my results? \\
\\
- What could not be done/expected? \\
\\
- What is still missing now? What would I do when continuing to work on the model? \\
\\

\subsection{From feature to label}

Linking the features to their designated label. \\
Combining the outcome of multiple networks for further processing.
