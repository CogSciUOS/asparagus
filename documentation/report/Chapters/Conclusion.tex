%----------------------------------------------------------------------------------------
%	CONCLUSION
%----------------------------------------------------------------------------------------
\section{Conclusion}
\label{ch:Conclusion}

In the scope of our project, we successfully adopted various classical as well as deep learning based computer vision approaches to classify asparagus spears based on their descriptive features or class labels. After collecting images, labeling, and processing the data, different trainable models were implemented from the fields of supervised learning, unsupervised learning and semi-supervised learning.

\bigskip
We could prove that computer vision and machine learning based techniques are practicable for the asparagus classification problem as they can be performed in a suitable time and our exploratory analysis yields promising results.

Our explorative study gave the possibility of a better overview on which techniques seem promising for asparagus classification and which might not need to be pursued in the future. Considering the amount of labeled data that is available, unsupervised and semi-supervised approaches gave some insight into the data but, in the end, they were more effortful to implement while not promising much better results than the supervised approaches. If there had been less labeled data, these approaches might have been more useful than approaches relying on labeled data.

Whether we have succeeded in improving the currently running sorting algorithm can not be said, yet. In cooperation with the local asparagus farm Gut Holsterfeld and the manufacturer of the asparagus sorting machine Autoselect ATS II, an implementation into the existing system can take place now and a method for evaluation can be defined.


\bigskip
Due to the direct start of the harvesting season at the beginning of the project, the sorting machine and the hardware setup had to be used as available (see also \autoref{sec:DiscussionMethodology}). However, possible improvements were discussed with the manufacturer of the machine.

A second camera to capture the head of the asparagus is needed in particular\footnote{This is already the case for newer versions of the Autoselect ATS II like the one at Querdel’s Hof.} which is reflected in our results. Especially for features like flower or rusty head, an additional head camera helps greatly with the classification (as shown in \nameref{subsec:HeadNetwork}). Another camera taking an image from the bottom of the spear could improve the detection of the feature hollow.\footnote{These cameras are already part of new asparagus sorting hardware like the one mentioned here: \url{https://www.neubauer-automation.de/uk/asparagus-sorting-machine-espaso-technicaldata.php} (visited on 04/28/2020)} Additional perspectives of the whole asparagus spears provide more information to determine certain features more accurately. For example, differences in lighting and reflection can carry useful information about the shape.

As mentioned in the \nameref{ch:Discussion}, the most crucial component for improving future asparagus sorting with the Autoselect ATS II is a further development of the setup. To improve the setup, it is conceivable that other sensor systems, such as laser technology to measure the exact size, could help to find relevant properties for asparagus classification \citep{bhargava2018fruits}. As this requires hardware changes to the sorting machine it is beyond the scope of our project.

\bigskip
Another claim that we started to address in the \nameref{sec:DiscussionMethodology}
is that the asparagus classification varies between different farms. Further, it also varies within one farm throughout the harvesting season. For example, one request of the farmer of Gut Holsterfeld is to have the possibility to sort asparagus in a better category if weather conditions reduce the usual amount of high quality asparagus. In order to meet this requirement, manual adjustment of parameters and a smooth transition for several features is needed. At the moment this is impossible with most contemporary neural networks. Since there is the temporal, sequential component of the harvesting season, it may be worthwhile to consider \acrshortpl{lstm} in combination with \acrshortpl{cnn}. According to the literature, there is still a wide range of possibilities, for example applying bayesian networks learning.

\bigskip
If this work will be followed by another project, the collected images, the different datasets and also the hand label app can be used for further approaches. Access to all images and the datasets can be gained through the university server.\footnote{ Until 01/31/2021, it can be accessed via \\ /net/projects/scratch/summer/valid\textunderscore until\textunderscore 31\textunderscore January\textunderscore 2021/jzerbe.} The documentation of the different datasets and the hand label app can be found in this report. The code can be found in our github repository.\footnote{ at \url{https://github.com/CogSciUOS/asparagus} (as of 11/27/2020)} A time-consuming part of our project was to collect data and explore different computer vision and machine learning approaches. A follow-up project could directly focus on developing and evaluating end-to-end pipelines for asparagus classification. Our results suggest using a binary feedforward \acrshortpl{cnn} would be a promising starting point for future work, because we have enough labeled data, and they offer a flexible and simple solution. Additional fine-tuning of the architecture to fit the individual needs of the to-be-predicted features (or class labels) and further preprocessing of the data will improve the network (as demonstrated in \autoref{subsec:FeatureEngineering}). The exploratory approaches from this project should be looked at for inspiration, but the unsupervised and semi-supervised approaches are not suggested to be continued. If additional labeled data is wanted, the hand label app can be used.\footnote{The code can be found at our Github repository at \url{https://github.com/CogSciUOS/asparagus/tree/FinalProject/labeling/hand\textunderscore label\textunderscore assistant} (as of 11/27/2020).} To finalize the project, the finished algorithm should be integrated into the sorting machine and evaluated in the real life setting.

\bigskip
In conclusion, we can confirm that modern approaches from computer vision and machine learning bear huge potential for the improvement of asparagus classification. We demonstrated that the algorithms we selected and adapted from scientific purposes could also be used in industrial applications. We strongly believe, machine learning approaches can help to improve the classification of asparagus into commercial quality classes.