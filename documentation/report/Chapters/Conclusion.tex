%----------------------------------------------------------------------------------------
%	CONCLUSION
%----------------------------------------------------------------------------------------
\section{Conclusion}
\label{ch:Conclusion}

In the scope of our project, we sucessfully adopted various classical as well as deep learning based computer vision approaches to classify asparagus spears according to their descriptive features or class labels. After collecting images, labeling, and processing the data, different trainable models were implemented from the fields of supervised learning, unsupervised learning and semi-supervised learning.

\bigskip
We could prove that computer vision and machine learning based techniques are practicable for the asparagus classification problem.

Our explorative study gave the possibility of a better overview on which techniques seem promising for asparagus classification and which might not need to be pursued in the future. As a next step, we would concentrate on using binary feedforward \acrshortpl{cnn} because they offer a flexible and simple solution. Additional fine-tuning of the architecture to fit the individual needs of the to-be-predicted features (or class labels) and further preprocessing of the data will improve the network (as demonstrated in \autoref{subsec:FeatureEngineering}). Considering the amount of labeled data that is available, unsupervised and semi-supervised approaches gave some insight into the data but, in the end, they were more effortful to implement while not promising much better results. If there had been less labeled data, these approaches might have been more useful than approaches relying on labeled data.

Whether we have succeeded in improving the currently running sorting algorithm can not be said, yet. In cooperation with the local asparagus farm Gut Holsterfeld and the manufacturer of the asparagus sorting machine Autoselect ATS II, a method for evaluation can now be developed. 


\bigskip
Due to the direct start of the harvesting season at the beginning of the project, the sorting machine and the hardware setup had to be used as available (see also \autoref{sec:DiscussionMethodology}). However, improvements were discussed with the manufacturer of the machine.

A second camera to capture the head of the asparagus is needed in particular\footnote{This is already the case for newer versions of the Autoselect ATS II like the one at Querdel’s Hof.} which is reflected in our results. Especially for features like flower or rusty head, an additional head camera helps greatly with the classification (as shown in \nameref{subsec:HeadNetwork}). Another camera taking an image from the bottom of the spear could improve the detection of the feature hollow.\footnote{These cameras are already part of new asparagus sorting hardware like the one mentioned here: \url{https://www.neubauer-automation.de/uk/asparagus-sorting-machine-espaso-technicaldata.php}}Additional perspectives of the hole asparagus spears give more usable information to determine certain features more accurately. For example differences in lighting and reflection can carry useful information about the shape. 

To further improve the setup, it is also conceivable that other sensor systems, such as laser technology, could help to find relevant properties for asparagus classification \citep{bhargava2018fruits}. As mentioned in the \nameref{ch:Discussion}, the most crucial component for improving future asparagus sorting with the Autoselect ATS II is a further development of the setup. This requires hardware changes to the sorting machine and is therefore beyond the scope of our project.

\bigskip
Another claim that we started to address in the \nameref{sec:DiscussionMethodology}
is that the asparagus classification varies between different farms. Further it also varies within one farm throughout the harvesting season. For example, one request of the farmer of Gut Holsterfeld is to have the possibility to sort asparagus in a higher category if weather conditions reduce the usual amount of high quality asparagus. In order to meet this requirement, manual adjustment of parameters and a smooth transition for several features is needed. At the moment this is impossible with most contemporary neural networks. Since there is the temporal, sequential component of the harvesting season, it may be worthwhile to consider \acrshortpl{lstm} in combination with  \acrshortpl{cnn}. According to the literature there are still many wide possibilities, for example applying bayesian networks learning.


\bigskip
In conclusion, we can confirm that modern approaches from computer vision and machine learning bear huge potential for the improvement of asparagus classification. Effective means of agile development allow for efficient collaboration in the production of the respective implementations. We demonstrated that the algorithms we selected could be used not only for scientific purposes but also in industrial applications. We strongly believe, machine learning approaches can help to improve the classification of asparagus into commercial quality classes.