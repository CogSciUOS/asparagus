%----------------------------------------------------------------------------------------
%	CONCLUSION
%----------------------------------------------------------------------------------------
\section{Conclusion and outlook}
\label{ch:Conclusion}

In this project, we successfully adopted various modern computer vision techniques with the aim of classifying asparagus spears according to their descriptive features or class labels. Alongside the development of the respective approaches, we examined the practicability in commercial applications. The specifications and goals of the project that we defined in the beginning helped structuring the agile planning sessions. As a result we implemented  several approaches that allow us to classify asparagus spears in dependence of their features or class label. After a time-consuming data collection, laborious labeling and classical pre-processing steps, different trainable models were implemented from the fields of supervised learning, unsupervised learning and semi-supervised learning.

Whether we have succeeded in improving the currently running sorting algorithm could not yet be clarified systematically because of the lack of time and resources for a suitable evaluation. In cooperation with the local asparagus farm Gut Holsterfeld and the manufacturer of the asparagus sorting machine Autoselect ATS II, a method can now be developed. However, we could prove that computer vision and machine learning based techniques are able to solve the asparagus classification problem.

The intention of the study project regarding the training and evaluation of project management~\citep{ studyregulations,moduledescription} has been followed up. Good self-organisation and team building became essential for the progress of the project. We have learned to organize, work as a team, put theoretical knowledge into practice, develop software, analyze data, and to practice the analysis and interpretation of statistical data under the conditions of a research project.

\bigskip
Due to the direct start of the harvesting season at the beginning of the project, the sorting machine and the hardware setup had to be used as already implemented. However, improvements were discussed with the manufacturer of the machine, Mr. Hermeler. A second camera to capture the head of the asparagus is needed in particular. This is reflected in our results. Especially for features like flower or rust, an additional head camera helps greatly with the classification. Another camera taking an image from the bottom of the spear could improve the detection of the feature hollow. These cameras are already implemented in some new asparagus sorting hardware.\footnote{see \\ \url{https://www.neubauer-automation.de/uk/asparagus-sorting-machine-espaso-technicaldata.php}}
Overall, it is an advantage to have even more perspectives on the asparagus, as some features might only be visible from a certain side. Namely, the lighting and reflection are slightly different, which can alter the color values, and distort the images. Hence, the asparagus might be stretched depending on the perspective. This brings in more usable information for classical computer vision algorithms to determine certain features accurately. 
It is also conceivable that other sensor systems, such as laser technology, could help to assign properties to asparagus. As mentioned in the~\nameref{ch:Discussion}, the most crucial step for archiving the goal to improve the asparagus sorting is the setup. We therefore have to adapt the preprocessing pipeline and fine-tune our approaches accordingly.

In order to put the results of this project into practice, a control system for the machine has to be implemented. For this purpose, a cooperation with the manufacturing company has to be arranged. After we developed results in the project, this can be the next step in the near future.

If our algorithm controls the sorting machine, it will be possible to create an evaluation. During our meetings we discussed existing difficulties of evaluation. One method is to measure and compare the sorting of the harvesters. Further methods need to be considered. If necessary, the setup for automated procedures can be extended.

Furthermore, it is possible to maintain data collection locally and use it for further improvement. An automated implementation, running locally on the computer of the machine is viable. Moreover locally generated networks could be collected and compared. Furthermore, merging then the different networks of different farms would be conceivable.

Another claim that we started to address in the~\nameref{sec:DiscussionMethodology}
is that the asparagus classification varies between farmers, as well as within one farm throughout the harvesting season. For example, upon request, asparagus should be placed in a higher category than normal after an unfavourable weather situation. In order to meet this requirement, manual adjustment of the screws and a smooth transition for several features is needed, which is impossible with most neural networks in their current form. Since we introduce a temporal, sequential component, it may be worthwhile to consider \acrshortpl{lstm} in combination with  \acrshortpl{cnn}. According to the literature there are still many wide possibilities, for example applying bayesian networks learning.

\bigskip
In conclusion, we can confirm that modern approaches from computer vision and machine learning bear potentials for the improvement of asparagus classification. Effective means of agile development allow for efficient collaboration in the production of the respective implementations. We demonstrated that the algorithms we selected could not only be used for scientific purposes but also in industrial applications. Modern machine learning approaches can help to improve solutions for the asparagus sorting problem.